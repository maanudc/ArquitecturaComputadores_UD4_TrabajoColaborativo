\documentclass[11pt,a4paper]{article}

\usepackage[margin=2cm]{geometry}
\usepackage{fontspec}
\usepackage{fancyhdr}
\usepackage{setspace}
\usepackage{tocloft}
\usepackage{microtype}
\usepackage{amssymb}
\usepackage{textcomp}
\usepackage{listings}
\usepackage{xcolor}
\usepackage{float}
\usepackage{booktabs}
\usepackage{needspace}
\usepackage{array}
\usepackage{amsmath}
\usepackage{tikz}
\usetikzlibrary{matrix,positioning,fit,backgrounds}
% === COLORES PREDEFINIDOS PARA AGRUPACIONES ===
\definecolor{kgroup1}{RGB}{255,200,200}  % Rojo claro
\definecolor{kgroup2}{RGB}{200,255,200}  % Verde claro
\definecolor{kgroup3}{RGB}{200,200,255}  % Azul claro
\definecolor{kgroup4}{RGB}{255,255,180}  % Amarillo claro
\definecolor{kgroup5}{RGB}{255,200,255}  % Magenta claro
\definecolor{kgroup6}{RGB}{200,255,255}  % Cian claro

% === MACRO PRINCIPAL PARA KARNAUGH DE 4 VARIABLES ===
% Uso: \karnaughCuatro{AB}{CD}{v0,v1,...,v15}{fila/col/color, ...}
\newcommand{\karnaughCuatro}[4]{%
\begin{tikzpicture}[scale=0.8]
  \def\cellsize{1.2}
  
  % Dibujar celdas con colores de fondo PRIMERO
  \foreach \row/\col/\bgcolor in {#4} {
    \fill[\bgcolor] (\col*\cellsize, -\row*\cellsize) rectangle ++(\cellsize,-\cellsize);
  }
  
  % Grid principal
  \draw[thick] (0,0) rectangle (4*\cellsize,-4*\cellsize);
  \draw (0,-\cellsize) -- (4*\cellsize,-\cellsize);
  \draw (0,-2*\cellsize) -- (4*\cellsize,-2*\cellsize);
  \draw (0,-3*\cellsize) -- (4*\cellsize,-3*\cellsize);
  \draw (\cellsize,0) -- (\cellsize,-4*\cellsize);
  \draw (2*\cellsize,0) -- (2*\cellsize,-4*\cellsize);
  \draw (3*\cellsize,0) -- (3*\cellsize,-4*\cellsize);
  
  % Etiqueta de variables (filas)
  \node[font=\bfseries] at (-0.6, -2*\cellsize) {#1};
  % Etiqueta de variables (columnas)
  \node[font=\bfseries] at (2*\cellsize, 0.6) {#2};
  
  % Código Gray para columnas (CD)
  \node at (0.5*\cellsize, 0.25) {00};
  \node at (1.5*\cellsize, 0.25) {01};
  \node at (2.5*\cellsize, 0.25) {11};
  \node at (3.5*\cellsize, 0.25) {10};
  
  % Código Gray para filas (AB)
  \node at (-0.35, -0.5*\cellsize) {00};
  \node at (-0.35, -1.5*\cellsize) {01};
  \node at (-0.35, -2.5*\cellsize) {11};
  \node at (-0.35, -3.5*\cellsize) {10};
  
  % Colocar valores en las celdas
  \foreach \val [count=\i from 0] in {#3} {
    \pgfmathtruncatemacro{\row}{int(\i/4)}
    \pgfmathtruncatemacro{\col}{mod(\i,4)}
    \node at (\col*\cellsize + 0.5*\cellsize, -\row*\cellsize - 0.5*\cellsize) {\val};
  }
\end{tikzpicture}%
}

% === MACRO SIMPLIFICADA SIN COLORES ===
\newcommand{\karnaughSimple}[3]{%
  \karnaughCuatro{#1}{#2}{#3}{}%
}

\lstset{
    language=C,
    basicstyle=\ttfamily\small,
    keywordstyle=\color{blue},
    commentstyle=\color{gray},
    stringstyle=\color{red},
    breaklines=true,
    showstringspaces=false,
    tabsize=4,
    frame=single,
    backgroundcolor=\color{white},
    aboveskip=6pt,
    belowskip=6pt
}

% Configuración de fuente: Verdana 11pt para todo el documento
\setmainfont[
  BoldFont=Verdana Bold.ttf,
  ItalicFont=Verdana Italic.ttf,
  BoldItalicFont=Verdana Bold Italic.ttf,
  Path=/System/Library/Fonts/Supplemental/
]{Verdana.ttf}

\usepackage[spanish,es-tabla]{babel}
\setstretch{1.5}

\usepackage{graphicx}
\renewcommand{\arraystretch}{1.25}
\setlength{\intextsep}{14pt plus 2pt minus 2pt}
\setlength{\textfloatsep}{18pt plus 2pt minus 4pt}
\setlength{\abovecaptionskip}{10pt}
\setlength{\belowcaptionskip}{6pt}

\usepackage{csquotes}
\usepackage[hidelinks]{hyperref}

\newcommand{\Concepto}{\textbf{Álgebra de Boole}}
\newcommand{\TituloTrabajo}{\Concepto}
\newcommand{\cmd}{{\fontspec{Apple Symbols}⌘}}

% Ambiente personalizado para listados normalizados
\lstnewenvironment{juanListado}[1]
{%
    \lstset{%
        caption={#1},
        aboveskip=10pt,
        belowskip=10pt,
        basicstyle={\setstretch{0.9}\ttfamily\footnotesize},
        captionpos=b,
        abovecaptionskip=15pt,
        belowcaptionskip=15pt
    }%
}
{}

% Ambiente personalizado para listados normalizados sin saltos de página
\lstnewenvironment{juanListadoNoBreaks}[1]
{%
    \needspace{10\baselineskip}%
    \lstset{%
        caption={#1},
        aboveskip=10pt,
        belowskip=10pt,
        basicstyle={\setstretch{0.9}\ttfamily\footnotesize},
        captionpos=b,
        abovecaptionskip=15pt,
        belowcaptionskip=15pt
    }%
}
{}

% 4 parámetros: num_entradas, num_salidas, contenido completo, caption
\newcommand{\tverdad}[4]{%
  \begin{table}[H]%
  \centering%
  \begin{tabular}{|*{#1}{c|}|*{#2}{c|}}%
    #3%
  \end{tabular}%
  \caption{#4}%
  \end{table}%
}

% Compilación recomendada (XeLaTeX):
% xelatex documento.tex
% biber documento
% xelatex documento.tex
% xelatex documento.tex

\title{Trabajo colaborativo UD4 Grupo 2}
\author{Manuel del Campo Abril\\Igor Lambarri Ostolaza\\Carlos Rodríguez San Miguel\\Juan Manuel Solsona Sagrado}
\date{\today}

\begin{document}
\maketitle

\pagenumbering{roman}

\tableofcontents
\newpage

\listoffigures
\newpage

\pagenumbering{arabic}
\setcounter{page}{1}

\section{Parte I}

\textbf{Enunciado:}

El objetivo es diseñar un circuito combinacional en el que reciba como entrada un número binario, que llamaremos N, de 4 bits (A,B,C,D) y tenga tres salidas $S_1$, $S_2$ y $S_3$. El circuito realizará las siguientes operaciones:
\begin{enumerate}
    \item Si $N \leq 5$, se activa la salida $S_1$, que encendería una luz verde.
    \item Si $5 < N \leq 10$, se activa la salida $S_3$, que encendería una luz ámbar.
    \item Si $N > 10$, se activa la salida $S_2$, que encendería una luz roja.
\end{enumerate}

Se pide:
\begin{enumerate}
    \item La tabla de verdad de cada una de las tres salidas.
    \item Representación numérica de cada función de las 3 salidas en suma de productos y en producto de sumas.
    \item Aplicar mapas de Karnaugh para simplificación de $S_1$ en suma de productos.
    \item Aplicar mapas de Karnaugh para simplificación de $S_2$ en suma de productos.
    \item Aplicar mapas de Karnaugh para simplificación de $S_3$ en suma de productos.
    \item Supón que tras la simplificación de una de las funciones, se obtiene $S=\bar{A}+BC$. Modifícala para poderla expresar de tal forma que únicamente se implemente con puertas NAND (conversión a puertas NAND).
    \item Implementar las tres expresiones mínimas obtenidas en los apartados 3, 4 y 5 mediante un circuito con puertas OR, AND y NOT.
    \item Implementar el mismo circuito en el siguiente simulador: \url{http://weblidi.info.unlp.edu.ar/catedras/organiza/circuitos/editor_simple.html}. Adjuntar pantallazos o enviar un enlace a un vídeo donde se explique qué se ha hecho y se visualice el circuito y la tabla de verdad.
\end{enumerate}

\textbf{Desarrollo}

\subsection*{Desarrollo de $S_1$}
\subsubsection*{Tabla de verdad de $S_1$}
Para poder detectar en A,B,C,D los valores $N \leq 5$ necesitamos definir su tabla de verdad:
\tverdad{6}{1}{%
  \hline
  DEC & A & B & C & D & $S_1$ \\
  \hline
  0  & 0 & 0 & 0 & 0 & 1 \\
  1  & 0 & 0 & 0 & 1 & 1 \\
  2  & 0 & 0 & 1 & 0 & 1 \\
  3  & 0 & 0 & 1 & 1 & 1 \\
  4  & 0 & 1 & 0 & 0 & 1 \\
  5  & 0 & 1 & 0 & 1 & 1 \\
  6  & 0 & 1 & 1 & 0 & 0 \\
  7  & 0 & 1 & 1 & 1 & 0 \\
  8  & 1 & 0 & 0 & 0 & 0 \\
  9  & 1 & 0 & 0 & 1 & 0 \\
  10 & 1 & 0 & 1 & 0 & 0 \\
  11 & 1 & 0 & 1 & 1 & 0 \\
  12 & 1 & 1 & 0 & 0 & 0 \\
  13 & 1 & 1 & 0 & 1 & 0 \\
  14 & 1 & 1 & 1 & 0 & 0 \\
  15 & 1 & 1 & 1 & 1 & 0 \\
  \hline%
}{Tabla de verdad para $S_1$}

\subsubsection*{Suma de productos de $S_1$}
Lo cual nos da una representación como suma de productos (minterms donde $S_1 = 1$):
\begin{align}
S_1 &= \bar{A}\bar{B}\bar{C}\bar{D} + \bar{A}\bar{B}\bar{C}D + \bar{A}\bar{B}C\bar{D} + \bar{A}\bar{B}CD + \bar{A}B\bar{C}\bar{D} + \bar{A}B\bar{C}D
\end{align}
O representado matemáticamente:
\[S_1 = \sum m(0,1,2,3,4,5)\]

\subsubsection*{Producto de sumas de $S_1$}
Para hallar el producto de sumas, identificamos los valores donde $S_1 = 0$ (el complemento), negamos cada variable según su valor en la tabla y construimos el productorio de maxterms.

\textbf{Universo complementario:} Los índices donde $S_1 = 0$ son $\{6,7,8,9,10,11,12,13,14,15\}$.

Para cada maxterm, negamos la variable si su bit es 1 y la dejamos sin negar si es 0:

\begin{center}
\begin{tabular}{|c|c|c|c|c|l|}
\hline
\textbf{Índice} & A & B & C & D & \textbf{Maxterm} \\
\hline
6  & 0 & 1 & 1 & 0 & $(A + \bar{B} + \bar{C} + D)$ \\
7  & 0 & 1 & 1 & 1 & $(A + \bar{B} + \bar{C} + \bar{D})$ \\
8  & 1 & 0 & 0 & 0 & $(\bar{A} + B + C + D)$ \\
9  & 1 & 0 & 0 & 1 & $(\bar{A} + B + C + \bar{D})$ \\
10 & 1 & 0 & 1 & 0 & $(\bar{A} + B + \bar{C} + D)$ \\
11 & 1 & 0 & 1 & 1 & $(\bar{A} + B + \bar{C} + \bar{D})$ \\
12 & 1 & 1 & 0 & 0 & $(\bar{A} + \bar{B} + C + D)$ \\
13 & 1 & 1 & 0 & 1 & $(\bar{A} + \bar{B} + C + \bar{D})$ \\
14 & 1 & 1 & 1 & 0 & $(\bar{A} + \bar{B} + \bar{C} + D)$ \\
15 & 1 & 1 & 1 & 1 & $(\bar{A} + \bar{B} + \bar{C} + \bar{D})$ \\
\hline
\end{tabular}
\end{center}

Por tanto, el producto de sumas queda:
\[S_1 = \prod M(6,7,8,9,10,11,12,13,14,15)\]

Expresado de forma expandida:
\begin{align*}
S_1 &= (A + \bar{B} + \bar{C} + D)(A + \bar{B} + \bar{C} + \bar{D}) \\
&\quad \cdot (\bar{A} + B + C + D)(\bar{A} + B + C + \bar{D}) \\
&\quad \cdot (\bar{A} + B + \bar{C} + D)(\bar{A} + B + \bar{C} + \bar{D}) \\
&\quad \cdot (\bar{A} + \bar{B} + C + D)(\bar{A} + \bar{B} + C + \bar{D}) \\
&\quad \cdot (\bar{A} + \bar{B} + \bar{C} + D)(\bar{A} + \bar{B} + \bar{C} + \bar{D})
\end{align*}

\subsubsection*{Diagrama de Karnaugh de $S_1$}

El diagrama de Karnaugh representa los valores de $S_1$ en orden Gray. Recordemos que el orden de las columnas es 00, 01, 11, 10 (no binario secuencial):

\begin{figure}[H]
\centering
\karnaughSimple{AB}{CD}{1,1,1,1, 1,1,0,0, 0,0,0,0, 0,0,0,0}
\caption{Mapa de Karnaugh para $S_1$ sin agrupar}
\end{figure}

\subsubsection*{Identificación de grupos}

Buscamos los grupos más grandes posibles de 1s adyacentes (potencias de 2):

\begin{figure}[H]
\centering
\karnaughCuatro{AB}{CD}%
  {1,1,1,1, 1,1,0,0, 0,0,0,0, 0,0,0,0}%
  {0/0/kgroup1, 0/1/kgroup1, 0/2/kgroup1, 0/3/kgroup1}
\caption{Grupo 1: $\bar{A}\bar{B}$ (4 celdas, fila AB=00)}
\end{figure}

\begin{figure}[H]
\centering
\karnaughCuatro{AB}{CD}%
  {1,1,1,1, 1,1,0,0, 0,0,0,0, 0,0,0,0}%
  {0/0/kgroup1, 0/1/kgroup1, 0/2/kgroup1, 0/3/kgroup1,
   1/0/kgroup2, 1/1/kgroup2}
\caption{Grupos identificados: Grupo 1 (rojo) = $\bar{A}\bar{B}$, Grupo 2 (verde) = $\bar{A}B\bar{C}$}
\end{figure}

\subsubsection*{Análisis de los grupos}

\begin{center}
\begin{tabular}{|c|c|c|c|}
\hline
\textbf{Grupo} & \textbf{Celdas (índices)} & \textbf{Variables constantes} & \textbf{Término} \\
\hline
1 (rojo) & 0, 1, 2, 3 & A=0, B=0 & $\bar{A}\bar{B}$ \\
\hline
2 (verde) & 4, 5 & A=0, B=1, C=0 & $\bar{A}B\bar{C}$ \\
\hline
\end{tabular}
\end{center}

\textbf{Explicación del proceso:}
\begin{itemize}
    \item \textbf{Grupo 1 (4 celdas):} Cubre toda la fila AB=00. Las variables C y D cambian de valor dentro del grupo, por lo que se eliminan. Solo A=0 y B=0 permanecen constantes, dando el término $\bar{A}\bar{B}$.
    \item \textbf{Grupo 2 (2 celdas):} Cubre las posiciones donde AB=01 y CD=00,01. La variable D cambia, por lo que se elimina. Las variables constantes son A=0, B=1, C=0, dando el término $\bar{A}B\bar{C}$.
\end{itemize}

\subsubsection*{Expresión simplificada de $S_1$}

Sumando los términos de cada grupo:
\[
\boxed{S_1 = \bar{A}\bar{B} + \bar{A}B\bar{C}}
\]

Esta expresión puede simplificarse algebraicamente factorizando $\bar{A}$:
\begin{align*}
S_1 &= \bar{A}\bar{B} + \bar{A}B\bar{C} \\
    &= \bar{A}(\bar{B} + B\bar{C}) \\
    &= \bar{A}(\bar{B} + \bar{C}) \quad \text{(por absorción: } X + \bar{X}Y = X + Y\text{)}
\end{align*}

Por tanto, la expresión mínima es:
\[
\boxed{S_1 = \bar{A}(\bar{B} + \bar{C})}
\]

\subsubsection*{Verificación}

Comprobamos que la expresión simplificada produce los mismos resultados:

\begin{center}
\begin{tabular}{|c|c|c|c|c|c|c|}
\hline
\textbf{N} & A & B & C & D & $\bar{A}(\bar{B}+\bar{C})$ & ¿Correcto? \\
\hline
0  & 0 & 0 & 0 & 0 & $1 \cdot (1+1) = 1$ & \checkmark \\
1  & 0 & 0 & 0 & 1 & $1 \cdot (1+1) = 1$ & \checkmark \\
2  & 0 & 0 & 1 & 0 & $1 \cdot (1+0) = 1$ & \checkmark \\
3  & 0 & 0 & 1 & 1 & $1 \cdot (1+0) = 1$ & \checkmark \\
4  & 0 & 1 & 0 & 0 & $1 \cdot (0+1) = 1$ & \checkmark \\
5  & 0 & 1 & 0 & 1 & $1 \cdot (0+1) = 1$ & \checkmark \\
6  & 0 & 1 & 1 & 0 & $1 \cdot (0+0) = 0$ & \checkmark \\
7  & 0 & 1 & 1 & 1 & $1 \cdot (0+0) = 0$ & \checkmark \\
8  & 1 & 0 & 0 & 0 & $0 \cdot (1+1) = 0$ & \checkmark \\
\hline
\end{tabular}
\end{center}

La expresión $S_1 = \bar{A}(\bar{B} + \bar{C})$ es correcta y equivale a detectar $N \leq 5$.

\newpage

\subsection*{Desarrollo de $S_3$}
\subsubsection*{Tabla de verdad de $S_3$}
Para poder detectar en A,B,C,D los valores $5 < N \leq 10$ necesitamos definir su tabla de verdad:
\tverdad{6}{1}{%
  \hline
  DEC & A & B & C & D & $S_3$ \\
  \hline
  0  & 0 & 0 & 0 & 0 & 0 \\
  1  & 0 & 0 & 0 & 1 & 0 \\
  2  & 0 & 0 & 1 & 0 & 0 \\
  3  & 0 & 0 & 1 & 1 & 0 \\
  4  & 0 & 1 & 0 & 0 & 0 \\
  5  & 0 & 1 & 0 & 1 & 0 \\
  6  & 0 & 1 & 1 & 0 & 1 \\
  7  & 0 & 1 & 1 & 1 & 1 \\
  8  & 1 & 0 & 0 & 0 & 1 \\
  9  & 1 & 0 & 0 & 1 & 1 \\
  10 & 1 & 0 & 1 & 0 & 1 \\
  11 & 1 & 0 & 1 & 1 & 0 \\
  12 & 1 & 1 & 0 & 0 & 0 \\
  13 & 1 & 1 & 0 & 1 & 0 \\
  14 & 1 & 1 & 1 & 0 & 0 \\
  15 & 1 & 1 & 1 & 1 & 0 \\
  \hline%
}{Tabla de verdad para $S_3$}

\subsubsection*{Suma de productos de $S_3$}
Lo cual nos da una representación como suma de productos (minterms donde $S_3 = 1$):
\begin{align}
S_3 &= \bar{A}BC\bar{D} + \bar{A}BCD + A\bar{B}\bar{C}\bar{D} + A\bar{B}\bar{C}D + A\bar{B}C\bar{D}
\end{align}
O representado matemáticamente:
\[S_3 = \sum m(6,7,8,9,10)\]

\subsubsection*{Producto de sumas de $S_3$}
Para hallar el producto de sumas, identificamos los valores donde $S_3 = 0$ (el complemento), negamos cada variable según su valor en la tabla y construimos el productorio de maxterms.

\textbf{Universo complementario:} Los índices donde $S_3 = 0$ son $\{0,1,2,3,4,5,11,12,13,14,15\}$.

Para cada maxterm, negamos la variable si su bit es 1 y la dejamos sin negar si es 0:

\begin{center}
\begin{tabular}{|c|c|c|c|c|l|}
\hline
\textbf{Índice} & A & B & C & D & \textbf{Maxterm} \\
\hline
0  & 0 & 0 & 0 & 0 & $(A + B + C + D)$ \\
1  & 0 & 0 & 0 & 1 & $(A + B + C + \bar{D})$ \\
2  & 0 & 0 & 1 & 0 & $(A + B + \bar{C} + D)$ \\
3  & 0 & 0 & 1 & 1 & $(A + B + \bar{C} + \bar{D})$ \\
4  & 0 & 1 & 0 & 0 & $(A + \bar{B} + C + D)$ \\
5  & 0 & 1 & 0 & 1 & $(A + \bar{B} + C + \bar{D})$ \\
11 & 1 & 0 & 1 & 1 & $(\bar{A} + B + \bar{C} + \bar{D})$ \\
12 & 1 & 1 & 0 & 0 & $(\bar{A} + \bar{B} + C + D)$ \\
13 & 1 & 1 & 0 & 1 & $(\bar{A} + \bar{B} + C + \bar{D})$ \\
14 & 1 & 1 & 1 & 0 & $(\bar{A} + \bar{B} + \bar{C} + D)$ \\
15 & 1 & 1 & 1 & 1 & $(\bar{A} + \bar{B} + \bar{C} + \bar{D})$ \\
\hline
\end{tabular}
\end{center}

Por tanto, el producto de sumas queda:
\[S_3 = \prod M(0,1,2,3,4,5,11,12,13,14,15)\]

Expresado de forma expandida:
\begin{align*}
S_3 &= (A + B + C + D)(A + B + C + \bar{D}) \\
&\quad \cdot (A + B + \bar{C} + D)(A + B + \bar{C} + \bar{D}) \\
&\quad \cdot (A + \bar{B} + C + D)(A + \bar{B} + C + \bar{D}) \\
&\quad \cdot (\bar{A} + B + \bar{C} + \bar{D}) \\
&\quad \cdot (\bar{A} + \bar{B} + C + D)(\bar{A} + \bar{B} + C + \bar{D}) \\
&\quad \cdot (\bar{A} + \bar{B} + \bar{C} + D)(\bar{A} + \bar{B} + \bar{C} + \bar{D})
\end{align*}

\subsubsection*{Diagrama de Karnaugh de $S_3$}

El diagrama de Karnaugh representa los valores de $S_3$ en orden Gray. Recordemos que el orden de las columnas es 00, 01, 11, 10 (no binario secuencial):

\begin{figure}[H]
\centering
\karnaughSimple{AB}{CD}{0,0,0,0, 0,0,1,1, 0,0,0,0, 1,1,0,1}
\caption{Mapa de Karnaugh para $S_3$ sin agrupar}
\end{figure}

\subsubsection*{Identificación de grupos}

Buscamos los grupos más grandes posibles de 1s adyacentes (potencias de 2):

\begin{figure}[H]
\centering
\karnaughCuatro{AB}{CD}%
  {0,0,0,0, 0,0,1,1, 0,0,0,0, 1,1,0,1}%
  {3/0/kgroup1, 3/1/kgroup1}
\caption{Grupo 1: $A\bar{B}\bar{C}$ (2 celdas, AB=10, CD=00,01)}
\end{figure}

\begin{figure}[H]
\centering
\karnaughCuatro{AB}{CD}%
  {0,0,0,0, 0,0,1,1, 0,0,0,0, 1,1,0,1}%
  {1/2/kgroup3, 1/3/kgroup3}
\caption{Grupo 2: $\bar{A}BC$ (2 celdas, AB=01, CD=11,10)}
\end{figure}

\begin{figure}[H]
\centering
\karnaughCuatro{AB}{CD}%
  {0,0,0,0, 0,0,1,1, 0,0,0,0, 1,1,0,1}%
  {3/3/kgroup2}
\caption{Grupo 3: $A\bar{B}C\bar{D}$ (1 celda, AB=10, CD=10)}
\end{figure}

\begin{figure}[H]
\centering
\karnaughCuatro{AB}{CD}%
  {0,0,0,0, 0,0,1,1, 0,0,0,0, 1,1,0,1}%
  {3/0/kgroup1, 3/1/kgroup1, 1/2/kgroup3, 1/3/kgroup3, 3/3/kgroup2}
\caption{Todos los grupos: Rojo = $A\bar{B}\bar{C}$, Azul = $\bar{A}BC$, Verde = $A\bar{B}C\bar{D}$}
\end{figure}

\subsubsection*{Análisis de los grupos}

\begin{center}
\begin{tabular}{|c|c|c|c|}
\hline
\textbf{Grupo} & \textbf{Celdas (índices)} & \textbf{Variables constantes} & \textbf{Término} \\
\hline
1 (rojo) & 8, 9 & A=1, B=0, C=0 & $A\bar{B}\bar{C}$ \\
\hline
2 (azul) & 6, 7 & A=0, B=1, C=1 & $\bar{A}BC$ \\
\hline
3 (verde) & 10 & A=1, B=0, C=1, D=0 & $A\bar{B}C\bar{D}$ \\
\hline
\end{tabular}
\end{center}

\textbf{Explicación del proceso:}
\begin{itemize}
    \item \textbf{Grupo 1 (2 celdas):} Cubre las posiciones donde AB=10 y CD=00,01. La variable D cambia, por lo que se elimina. Las variables constantes son A=1, B=0, C=0, dando el término $A\bar{B}\bar{C}$.
    \item \textbf{Grupo 2 (2 celdas):} Cubre las posiciones donde AB=01 y CD=11,10. La variable D cambia, por lo que se elimina. Las variables constantes son A=0, B=1, C=1, dando el término $\bar{A}BC$.
    \item \textbf{Grupo 3 (1 celda):} Celda aislada en AB=10, CD=10. No se puede agrupar con otras celdas. Todas las variables son constantes: A=1, B=0, C=1, D=0, dando el término $A\bar{B}C\bar{D}$.
\end{itemize}

\subsubsection*{Expresión simplificada de $S_3$}

Sumando los términos de cada grupo:
\[
\boxed{S_3 = A\bar{B}\bar{C} + \bar{A}BC + A\bar{B}C\bar{D}}
\]

Esta expresión puede simplificarse parcialmente factorizando $A\bar{B}$:
\begin{align*}
S_3 &= A\bar{B}\bar{C} + \bar{A}BC + A\bar{B}C\bar{D} \\
    &= A\bar{B}(\bar{C} + C\bar{D}) + \bar{A}BC \\
    &= A\bar{B}(\bar{C} + \bar{D}) + \bar{A}BC \quad \text{(por absorción: } X + \bar{X}Y = X + Y\text{)}
\end{align*}

Por tanto, la expresión mínima es:
\[
\boxed{S_3 = A\bar{B}(\bar{C} + \bar{D}) + \bar{A}BC}
\]

\subsubsection*{Verificación}

Comprobamos que la expresión simplificada produce los mismos resultados para $5 < N \leq 10$:

\begin{center}
\begin{tabular}{|c|c|c|c|c|c|c|}
\hline
\textbf{N} & A & B & C & D & $A\bar{B}(\bar{C}+\bar{D}) + \bar{A}BC$ & ¿Correcto? \\
\hline
5  & 0 & 1 & 0 & 1 & $0 + 0 \cdot 1 \cdot 0 = 0$ & \checkmark \\
6  & 0 & 1 & 1 & 0 & $0 + 1 \cdot 1 \cdot 1 = 1$ & \checkmark \\
7  & 0 & 1 & 1 & 1 & $0 + 1 \cdot 1 \cdot 1 = 1$ & \checkmark \\
8  & 1 & 0 & 0 & 0 & $1 \cdot 1 \cdot (1+1) + 0 = 1$ & \checkmark \\
9  & 1 & 0 & 0 & 1 & $1 \cdot 1 \cdot (1+0) + 0 = 1$ & \checkmark \\
10 & 1 & 0 & 1 & 0 & $1 \cdot 1 \cdot (0+1) + 0 = 1$ & \checkmark \\
11 & 1 & 0 & 1 & 1 & $1 \cdot 1 \cdot (0+0) + 0 = 0$ & \checkmark \\
12 & 1 & 1 & 0 & 0 & $1 \cdot 0 \cdot (1+1) + 0 = 0$ & \checkmark \\
\hline
\end{tabular}
\end{center}

La expresión $S_3 = A\bar{B}(\bar{C} + \bar{D}) + \bar{A}BC$ es correcta y equivale a detectar $5 < N \leq 10$.

\newpage

\subsection*{Desarrollo de $S_2$}
\subsubsection*{Tabla de verdad de $S_2$}
Para poder detectar en A,B,C,D los valores $N > 10$ necesitamos definir su tabla de verdad:
\tverdad{6}{1}{%
  \hline
  DEC & A & B & C & D & $S_2$ \\
  \hline
  0  & 0 & 0 & 0 & 0 & 0 \\
  1  & 0 & 0 & 0 & 1 & 0 \\
  2  & 0 & 0 & 1 & 0 & 0 \\
  3  & 0 & 0 & 1 & 1 & 0 \\
  4  & 0 & 1 & 0 & 0 & 0 \\
  5  & 0 & 1 & 0 & 1 & 0 \\
  6  & 0 & 1 & 1 & 0 & 0 \\
  7  & 0 & 1 & 1 & 1 & 0 \\
  8  & 1 & 0 & 0 & 0 & 0 \\
  9  & 1 & 0 & 0 & 1 & 0 \\
  10 & 1 & 0 & 1 & 0 & 0 \\
  11 & 1 & 0 & 1 & 1 & 1 \\
  12 & 1 & 1 & 0 & 0 & 1 \\
  13 & 1 & 1 & 0 & 1 & 1 \\
  14 & 1 & 1 & 1 & 0 & 1 \\
  15 & 1 & 1 & 1 & 1 & 1 \\
  \hline%
}{Tabla de verdad para $S_2$}

\subsubsection*{Suma de productos de $S_2$}
Lo cual nos da una representación como suma de productos (minterms donde $S_2 = 1$):
\begin{align}
S_2 &= A\bar{B}CD + AB\bar{C}\bar{D} + AB\bar{C}D + ABC\bar{D} + ABCD
\end{align}
O representado matemáticamente:
\[S_2 = \sum m(11,12,13,14,15)\]

\subsubsection*{Producto de sumas de $S_2$}
Para hallar el producto de sumas, identificamos los valores donde $S_2 = 0$ (el complemento), negamos cada variable según su valor en la tabla y construimos el productorio de maxterms.

\textbf{Universo complementario:} Los índices donde $S_2 = 0$ son $\{0,1,2,3,4,5,6,7,8,9,10\}$.

Para cada maxterm, negamos la variable si su bit es 1 y la dejamos sin negar si es 0:

\begin{center}
\begin{tabular}{|c|c|c|c|c|l|}
\hline
\textbf{Índice} & A & B & C & D & \textbf{Maxterm} \\
\hline
0  & 0 & 0 & 0 & 0 & $(A + B + C + D)$ \\
1  & 0 & 0 & 0 & 1 & $(A + B + C + \bar{D})$ \\
2  & 0 & 0 & 1 & 0 & $(A + B + \bar{C} + D)$ \\
3  & 0 & 0 & 1 & 1 & $(A + B + \bar{C} + \bar{D})$ \\
4  & 0 & 1 & 0 & 0 & $(A + \bar{B} + C + D)$ \\
5  & 0 & 1 & 0 & 1 & $(A + \bar{B} + C + \bar{D})$ \\
6  & 0 & 1 & 1 & 0 & $(A + \bar{B} + \bar{C} + D)$ \\
7  & 0 & 1 & 1 & 1 & $(A + \bar{B} + \bar{C} + \bar{D})$ \\
8  & 1 & 0 & 0 & 0 & $(\bar{A} + B + C + D)$ \\
9  & 1 & 0 & 0 & 1 & $(\bar{A} + B + C + \bar{D})$ \\
10 & 1 & 0 & 1 & 0 & $(\bar{A} + B + \bar{C} + D)$ \\
\hline
\end{tabular}
\end{center}

Por tanto, el producto de sumas queda:
\[S_2 = \prod M(0,1,2,3,4,5,6,7,8,9,10)\]

Expresado de forma expandida:
\begin{align*}
S_2 &= (A + B + C + D)(A + B + C + \bar{D}) \\
&\quad \cdot (A + B + \bar{C} + D)(A + B + \bar{C} + \bar{D}) \\
&\quad \cdot (A + \bar{B} + C + D)(A + \bar{B} + C + \bar{D}) \\
&\quad \cdot (A + \bar{B} + \bar{C} + D)(A + \bar{B} + \bar{C} + \bar{D}) \\
&\quad \cdot (\bar{A} + B + C + D)(\bar{A} + B + C + \bar{D}) \\
&\quad \cdot (\bar{A} + B + \bar{C} + D)
\end{align*}

\subsubsection*{Diagrama de Karnaugh de $S_2$}

El diagrama de Karnaugh representa los valores de $S_2$ en orden Gray. Recordemos que el orden de las columnas es 00, 01, 11, 10 (no binario secuencial):

\begin{figure}[H]
\centering
\karnaughSimple{AB}{CD}{0,0,0,0, 0,0,0,0, 1,1,1,1, 0,0,1,0}
\caption{Mapa de Karnaugh para $S_2$ sin agrupar}
\end{figure}

\subsubsection*{Identificación de grupos}

Buscamos los grupos más grandes posibles de 1s adyacentes (potencias de 2):

\begin{figure}[H]
\centering
\karnaughCuatro{AB}{CD}%
  {0,0,0,0, 0,0,0,0, 1,1,1,1, 0,0,1,0}%
  {2/0/kgroup1, 2/1/kgroup1, 2/2/kgroup1, 2/3/kgroup1}
\caption{Grupo 1: $AB$ (4 celdas, fila AB=11 completa)}
\end{figure}

\begin{figure}[H]
\centering
\karnaughCuatro{AB}{CD}%
  {0,0,0,0, 0,0,0,0, 1,1,1,1, 0,0,1,0}%
  {2/2/kgroup2, 3/2/kgroup2}
\caption{Grupo 2: $ACD$ (2 celdas, CD=11, A=1)}
\end{figure}

\begin{figure}[H]
\centering
\karnaughCuatro{AB}{CD}%
  {0,0,0,0, 0,0,0,0, 1,1,1,1, 0,0,1,0}%
  {2/0/kgroup1, 2/1/kgroup1, 2/2/kgroup1, 2/3/kgroup1, 3/2/kgroup2}
\caption{Todos los grupos: Rojo = $AB$, Verde = $ACD$}
\end{figure}

\subsubsection*{Análisis de los grupos}

\begin{center}
\begin{tabular}{|c|c|c|c|}
\hline
\textbf{Grupo} & \textbf{Celdas (índices)} & \textbf{Variables constantes} & \textbf{Término} \\
\hline
1 (rojo) & 12, 13, 14, 15 & A=1, B=1 & $AB$ \\
\hline
2 (verde) & 11, 15 & A=1, C=1, D=1 & $ACD$ \\
\hline
\end{tabular}
\end{center}

\textbf{Explicación del proceso:}
\begin{itemize}
    \item \textbf{Grupo 1 (4 celdas):} Cubre toda la fila AB=11. Las variables C y D cambian de valor dentro del grupo, por lo que se eliminan. Solo A=1 y B=1 permanecen constantes, dando el término $AB$.
    \item \textbf{Grupo 2 (2 celdas):} Cubre las posiciones donde CD=11 y A=1 (índices 11 y 15). La variable B cambia, por lo que se elimina. Las variables constantes son A=1, C=1, D=1, dando el término $ACD$.
\end{itemize}

\subsubsection*{Expresión simplificada de $S_2$}

Sumando los términos de cada grupo:
\[
\boxed{S_2 = AB + ACD}
\]

Esta expresión puede simplificarse factorizando $A$:
\begin{align*}
S_2 &= AB + ACD \\
    &= A(B + CD)
\end{align*}

Por tanto, la expresión mínima es:
\[
\boxed{S_2 = A(B + CD)}
\]

\subsubsection*{Verificación}

Comprobamos que la expresión simplificada produce los mismos resultados para $N > 10$:

\begin{center}
\begin{tabular}{|c|c|c|c|c|c|c|}
\hline
\textbf{N} & A & B & C & D & $A(B + CD)$ & ¿Correcto? \\
\hline
9  & 1 & 0 & 0 & 1 & $1 \cdot (0 + 0) = 0$ & \checkmark \\
10 & 1 & 0 & 1 & 0 & $1 \cdot (0 + 0) = 0$ & \checkmark \\
11 & 1 & 0 & 1 & 1 & $1 \cdot (0 + 1) = 1$ & \checkmark \\
12 & 1 & 1 & 0 & 0 & $1 \cdot (1 + 0) = 1$ & \checkmark \\
13 & 1 & 1 & 0 & 1 & $1 \cdot (1 + 0) = 1$ & \checkmark \\
14 & 1 & 1 & 1 & 0 & $1 \cdot (1 + 0) = 1$ & \checkmark \\
15 & 1 & 1 & 1 & 1 & $1 \cdot (1 + 1) = 1$ & \checkmark \\
0  & 0 & 0 & 0 & 0 & $0 \cdot (0 + 0) = 0$ & \checkmark \\
\hline
\end{tabular}
\end{center}

La expresión $S_2 = A(B + CD)$ es correcta y equivale a detectar $N > 10$.

\newpage

\subsection*{Conversión a puertas NAND}

\subsubsection*{Enunciado}
Supón que tras la simplificación de una de las funciones, se obtiene $S = \bar{A} + BC$. Modifícala para poderla expresar de tal forma que únicamente se implemente con puertas NAND.

\subsubsection*{Desarrollo}

Aplicamos doble negación y De Morgan:
\begin{align*}
S &= \bar{A} + BC \\
  &= \overline{\overline{\bar{A} + BC}} \\
  &= \overline{\overline{\bar{A}} \cdot \overline{BC}} \\
  &= \overline{A \cdot \overline{BC}}
\end{align*}

\subsubsection*{Resultado}

\[
\boxed{S = \overline{A \cdot \overline{BC}} = \text{NAND}(A, \text{NAND}(B, C))}
\]

El circuito se implementa con \textbf{2 puertas NAND}.

\newpage

\subsection*{Implementación con puertas OR, AND y NOT}

Las tres expresiones mínimas obtenidas son:

\begin{center}
\begin{tabular}{|c|c|l|c|}
\hline
\textbf{Salida} & \textbf{Condición} & \textbf{Expresión mínima} & \textbf{Puertas} \\
\hline
$S_1$ & $N \leq 5$ & $\bar{A}(\bar{B} + \bar{C})$ & 3 NOT, 1 OR, 1 AND \\
\hline
$S_2$ & $N > 10$ & $A(B + CD)$ & 2 AND, 1 OR \\
\hline
$S_3$ & $5 < N \leq 10$ & $A\bar{B}(\bar{C} + \bar{D}) + \bar{A}BC$ & 4 NOT, 2 OR, 4 AND \\
\hline
\end{tabular}
\end{center}

\subsubsection*{Circuito para $S_1 = \bar{A}(\bar{B} + \bar{C})$}

\begin{figure}[H]
\centering
\includegraphics[width=0.6\textwidth]{imagenes/S1.png}
\caption{Circuito de $S_1$ con puertas NOT, OR y AND}
\end{figure}

\subsubsection*{Circuito para $S_2 = A(B + CD)$}

\begin{figure}[H]
\centering
\includegraphics[width=0.6\textwidth]{imagenes/S2.png}
\caption{Circuito de $S_2$ con puertas AND y OR}
\end{figure}

\subsubsection*{Circuito para $S_3 = A\bar{B}(\bar{C} + \bar{D}) + \bar{A}BC$}

\begin{figure}[H]
\centering
\includegraphics[width=0.7\textwidth]{imagenes/S3.png}
\caption{Circuito de $S_3$ con puertas NOT, OR y AND}
\end{figure}

\newpage

\subsection*{Implementación en simulador}

\newpage

\subsection*{Verificación en simulador}

Los circuitos no han sido implementados y verificados en el simulador:
\url{http://weblidi.info.unlp.edu.ar/catedras/organiza/circuitos/editor_simple.html}, sin embargo se ha usado:\url{https://logic.ly/demo/}


A continuación se muestran las 16 combinaciones posibles de entrada y su resultado esperado:


\begin{figure}[H]
\centering
\includegraphics[width=0.7\textwidth]{imagenes/SIM_0.jpg}
\caption{N=0 (A=0, B=0, C=0, D=0) — Esperado: $S_1$=1, $S_2$=0, $S_3$=0}
\end{figure}

\begin{figure}[H]
\centering
\includegraphics[width=0.7\textwidth]{imagenes/SIM_1.jpg}
\caption{N=1 (A=0, B=0, C=0, D=1) — Esperado: $S_1$=1, $S_2$=0, $S_3$=0}
\end{figure}

\begin{figure}[H]
\centering
\includegraphics[width=0.7\textwidth]{imagenes/SIM_2.jpg}
\caption{N=2 (A=0, B=0, C=1, D=0) — Esperado: $S_1$=1, $S_2$=0, $S_3$=0}
\end{figure}

\begin{figure}[H]
\centering
\includegraphics[width=0.7\textwidth]{imagenes/SIM_3.jpg}
\caption{N=3 (A=0, B=0, C=1, D=1) — Esperado: $S_1$=1, $S_2$=0, $S_3$=0}
\end{figure}

\begin{figure}[H]
\centering
\includegraphics[width=0.7\textwidth]{imagenes/SIM_4.jpg}
\caption{N=4 (A=0, B=1, C=0, D=0) — Esperado: $S_1$=1, $S_2$=0, $S_3$=0}
\end{figure}

\begin{figure}[H]
\centering
\includegraphics[width=0.7\textwidth]{imagenes/SIM_5.jpg}
\caption{N=5 (A=0, B=1, C=0, D=1) — Esperado: $S_1$=1, $S_2$=0, $S_3$=0}
\end{figure}

\begin{figure}[H]
\centering
\includegraphics[width=0.7\textwidth]{imagenes/SIM_6.jpg}
\caption{N=6 (A=0, B=1, C=1, D=0) — Esperado: $S_1$=0, $S_2$=0, $S_3$=1}
\end{figure}

\begin{figure}[H]
\centering
\includegraphics[width=0.7\textwidth]{imagenes/SIM_7.jpg}
\caption{N=7 (A=0, B=1, C=1, D=1) — Esperado: $S_1$=0, $S_2$=0, $S_3$=1}
\end{figure}

\begin{figure}[H]
\centering
\includegraphics[width=0.7\textwidth]{imagenes/SIM_8.jpg}
\caption{N=8 (A=1, B=0, C=0, D=0) — Esperado: $S_1$=0, $S_2$=0, $S_3$=1}
\end{figure}

\begin{figure}[H]
\centering
\includegraphics[width=0.7\textwidth]{imagenes/SIM_9.jpg}
\caption{N=9 (A=1, B=0, C=0, D=1) — Esperado: $S_1$=0, $S_2$=0, $S_3$=1}
\end{figure}

\begin{figure}[H]
\centering
\includegraphics[width=0.7\textwidth]{imagenes/SIM_10.jpg}
\caption{N=10 (A=1, B=0, C=1, D=0) — Esperado: $S_1$=0, $S_2$=0, $S_3$=1}
\end{figure}

\begin{figure}[H]
\centering
\includegraphics[width=0.7\textwidth]{imagenes/SIM_11.jpg}
\caption{N=11 (A=1, B=0, C=1, D=1) — Esperado: $S_1$=0, $S_2$=1, $S_3$=0}
\end{figure}

\begin{figure}[H]
\centering
\includegraphics[width=0.7\textwidth]{imagenes/SIM_12.jpg}
\caption{N=12 (A=1, B=1, C=0, D=0) — Esperado: $S_1$=0, $S_2$=1, $S_3$=0}
\end{figure}

\begin{figure}[H]
\centering
\includegraphics[width=0.7\textwidth]{imagenes/SIM_13.jpg}
\caption{N=13 (A=1, B=1, C=0, D=1) — Esperado: $S_1$=0, $S_2$=1, $S_3$=0}
\end{figure}

\begin{figure}[H]
\centering
\includegraphics[width=0.7\textwidth]{imagenes/SIM_14.jpg}
\caption{N=14 (A=1, B=1, C=1, D=0) — Esperado: $S_1$=0, $S_2$=1, $S_3$=0}
\end{figure}

\begin{figure}[H]
\centering
\includegraphics[width=0.7\textwidth]{imagenes/SIM_15.jpg}
\caption{N=15 (A=1, B=1, C=1, D=1) — Esperado: $S_1$=0, $S_2$=1, $S_3$=0}
\end{figure}
\end{document}