% === COLORES PREDEFINIDOS PARA AGRUPACIONES ===
\definecolor{kgroup1}{RGB}{255,200,200}  % Rojo claro
\definecolor{kgroup2}{RGB}{200,255,200}  % Verde claro
\definecolor{kgroup3}{RGB}{200,200,255}  % Azul claro
\definecolor{kgroup4}{RGB}{255,255,180}  % Amarillo claro
\definecolor{kgroup5}{RGB}{255,200,255}  % Magenta claro
\definecolor{kgroup6}{RGB}{200,255,255}  % Cian claro

% === MACRO PRINCIPAL PARA KARNAUGH DE 4 VARIABLES ===
% Uso: \karnaughCuatro{AB}{CD}{v0,v1,...,v15}{fila/col/color, ...}
\newcommand{\karnaughCuatro}[4]{%
\begin{tikzpicture}[scale=0.8]
  \def\cellsize{1.2}
  
  % Dibujar celdas con colores de fondo PRIMERO
  \foreach \row/\col/\bgcolor in {#4} {
    \fill[\bgcolor] (\col*\cellsize, -\row*\cellsize) rectangle ++(\cellsize,-\cellsize);
  }
  
  % Grid principal
  \draw[thick] (0,0) rectangle (4*\cellsize,-4*\cellsize);
  \draw (0,-\cellsize) -- (4*\cellsize,-\cellsize);
  \draw (0,-2*\cellsize) -- (4*\cellsize,-2*\cellsize);
  \draw (0,-3*\cellsize) -- (4*\cellsize,-3*\cellsize);
  \draw (\cellsize,0) -- (\cellsize,-4*\cellsize);
  \draw (2*\cellsize,0) -- (2*\cellsize,-4*\cellsize);
  \draw (3*\cellsize,0) -- (3*\cellsize,-4*\cellsize);
  
  % Etiqueta de variables (filas)
  \node[font=\bfseries] at (-0.6, -2*\cellsize) {#1};
  % Etiqueta de variables (columnas)
  \node[font=\bfseries] at (2*\cellsize, 0.6) {#2};
  
  % Código Gray para columnas (CD)
  \node at (0.5*\cellsize, 0.25) {00};
  \node at (1.5*\cellsize, 0.25) {01};
  \node at (2.5*\cellsize, 0.25) {11};
  \node at (3.5*\cellsize, 0.25) {10};
  
  % Código Gray para filas (AB)
  \node at (-0.35, -0.5*\cellsize) {00};
  \node at (-0.35, -1.5*\cellsize) {01};
  \node at (-0.35, -2.5*\cellsize) {11};
  \node at (-0.35, -3.5*\cellsize) {10};
  
  % Colocar valores en las celdas
  \foreach \val [count=\i from 0] in {#3} {
    \pgfmathtruncatemacro{\row}{int(\i/4)}
    \pgfmathtruncatemacro{\col}{mod(\i,4)}
    \node at (\col*\cellsize + 0.5*\cellsize, -\row*\cellsize - 0.5*\cellsize) {\val};
  }
\end{tikzpicture}%
}

% === MACRO SIMPLIFICADA SIN COLORES ===
\newcommand{\karnaughSimple}[3]{%
  \karnaughCuatro{#1}{#2}{#3}{}%
}